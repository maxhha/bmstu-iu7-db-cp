\chapter*{ВВЕДЕНИЕ}
\addcontentsline{toc}{chapter}{ВВЕДЕНИЕ}

% С продвижением прогресса появляется необходимость в инструментах, находящихся на пересечении нескольких наук. Одним из таких пересечений является медицина и компьютерная графика. С помощью магнитно-резонансной томографии и компьютерной томографии собирают информацию о внутренней структуре органов и тканей. Затем эти данные должны быть корректно отображены на экране компьютера, чтобы медицинский специалист смог поставить диагноз.

% В то же время, сейчас активно развиваются и применяются технологии трехмерной печати. В медицине они используются для создания протезов и имплантов. Индивидуально напечатанные протезы значительно увеличивают качество жизни. Но для их создания необходима информация о внутренней структуре заменяемого органа. Таким образом, возникает потребность в программном обеспечении, позволяющем по трехмерным снимкам получать файлы для трехмерной печати.

Аукционы набирают популярность: так в 2021 году количество торгов в России выросло на 17\% и объем продаж на 18\% по сравнению с 2020 годом\cite{artauctionstat2021}. Это показывает, что у россиян постепенно появляется интерес к теме торгов и аукционов.

Целью моей работы является разработка базы данных для проведения онлайн аукциона и реализация сервера, предоставляющего программный доступ к ней.

Для достижения поставленной цели необходимо выполнить следующие задачи:

\begin{itemize}
    \item изучить и проанализировать специфику проведения аукциона;
    \item проанализировать способы представления данных и выбрать оптимальный для поставленной задачи;
    \item описать используемые структуры данных;
    % \item оценить объем памяти, необходимый для хранения данных;
    \item описать структуру разрабатываемого программного обеспечения;
    \item привести основные сценарии использования приложения;
    \item определить средства программной реализации;
    \item описать процесс сборки приложения;
    \item протестировать разработанное программное обеспечение.
\end{itemize}
