\chapter{Аналитический раздел}
В данном разделе рассмотрены определение и  классификация аукционов, порядок проведения и правовой аспект английского аукциона. Также, представлены варианты использования приложения, формализованы данные, построена диаграмма сущность-связь и рассмотрены состояния сущностей систем. Проанализированы и классифицированы модели хранения данных, выбраны и обоснованы критерии сравнения моделей, и из рассмотренных моделей выбрана оптимальная для решения поставленной задачи.

\section{Постановка задачи}

Необходимо спроектировать и реализовать базу данных, для хранения всей необходимой информации для проведения английского аукциона по сети. База данных должна содержать информацию о товарах, предложениях и движениях денежных средств. Также, нужно разработать программный интерфейс, который позволит работать с этой базой данных: создавать, получать, удалять и редактировать информацию о товарах, предложениях и транзакциях.

\section{Определение аукциона}

Аукцион можно определить по одному из его основных свойств: механизму достижения рыночного равновесия, уравнивания спроса и предложений. Среди всех доступных механизмов достижения рыночного равновесия аукцион отличается явным ценообразованием: всем участвующим в аукционе сторонам заранее известны правила, определяющие конечную цену. На аукционе чаще всего продают объекты, для которых не существует установившегося рынка, то есть они или редкие, или уникальные. Например, на международных аукционах покупаются: пушно-меховые товары, немытая шерсть, подержанные автомобили, картины, лошади. \cite{menezes2004introduction}

\section{Классификация аукционов}

С точки зрения открытости ставок аукционы делятся на следующие виды:
\begin{itemize}
    \item \textbf{открытый аукцион} -- аукцион, в котором все ставки известны его участникам;
    \item \textbf{закрытый аукцион} -- аукцион, в котором ставки участников известны только организатору.
\end{itemize}

По способу установления цены аукционы делятся на следующие виды:
\begin{itemize}
    \item \textbf{аукцион с повышением цены} начинается с минимальной цены, а ставки должны увеличиваться. Покупателем является тот, кто предложил наибольшую ставку;
    \item \textbf{аукцион с понижением цены} начинается с максимальной цены, которая в ходе аукциона постепенно уменьшается. Покупателем является тот, кто первым согласился на предложенную цену.
\end{itemize}

\textbf{Английский аукцион} является классическим, наиболее часто применяемым, открытым аукционом с повышением цены.

\section{Стадии проведения аукциона}

Различают четыре стадии проведения аукционов \cite{strovskiy2017ved}:
\begin{itemize}
    \item \textbf{подготовка} -- на этом этапе владелец товара доставляет его на склад организатора аукциона, составляются каталоги, осуществляется рекламная деятельность;
    \item \textbf{осмотр товаров} -- во время осмотра товара потенциальные покупатели имеют возможность ознакомиться с выставленными для продажи товарами. Осмотр является важным этапом проведения аукционных торгов, так как в случае приобретения товара претензии к его качеству впоследствии не принимаются;
    \item \textbf{аукционный торг} -- главная стадия аукциона, на которой определяется стоимость товара
    \item \textbf{оформление и исполнение аукционной сделки}.
\end{itemize}

\section{Аукционный торг}

В случае английского аукциона торг проходит следующим образом \cite{strovskiy2017ved}:
\begin{enumerate}
    \item Аукционист объявляет начальную цену и спрашивает: "Кто больше?"{}
    \item Покупатель, желающий приобрести лот по более высокой цене, называет новую цену, которая выше предыдущей на величину не ниже минимальной надбавки, указанной в правилах проведения торгов.
    \item Аукционист называет номер покупателя, под которым он зарегистрирован на аукционе, новую цену лота и повторяет вопрос: "Кто больше?"{}
    \item Если после троекратного повторения вопроса не следует нового предложения, аукционист ударяет молотком, подтверждая продажу лота покупателю, который последним назвал наивысшую цену.
\end{enumerate}

\section{Оформление и исполнение аукционной сделки}

Согласно Гражданскому кодексу Российской Федерации по результатам торгов организатор и лицо, выигравшее торги, подписывают в день проведения аукциона или конкурса протокол о результатах торгов, который имеет силу договора. При заключении договора с лицом, выигравшим торги, сумма внесенного им перед началом торгов задатка, засчитывается в счет исполнения обязательств по заключенному договору. \cite{lawauction} 

Для совершения операций с денежными средствами, права на которые принадлежат другому лицу-бенефициару, в Гражданском кодексе Российской Федерации предусмотрен номинальный счет \cite{lawnominalaccount}. Номинальный счет формально имеет одного владельца, а на нем лежат денежные средства, права на которые принадлежат другому лицу. По завершению аукционной сделки, с помощью банка, должна происходить передача прав на денежные средства от лица, выигравшего торги, к владельцу предмета торгов и организатору.

\section{Варианты использования}

В аукционе можно выделить следующих участников:
\begin{itemize}
    \item \textbf{владелец} -- пользователь, который владеет некоторым товаром и имеет желание его продать через аукцион;
    \item \textbf{покупатель} -- пользователь, который хочет приобрести товар на аукционе;
    \item \textbf{менеджер} -- представитель компании, контролирующий правильность заполнения анкет, информации о продуктах и управляющий номинальными счетами аукционной платформы.
\end{itemize}

На рисунке \ref{img:usecase} представлена диаграмма вариантов использования приложения каждым из участников.

\imgs{usecase}{th}{0.8}{Диаграмма вариантов использования}

\section{Формализация данных}

База данных должна хранить информацию о:
\begin{itemize}
    \item пользователях;
    \item товарах;
    \item аукцион;
    \item ставках;
    \item заключенных договорах;
    \item номинальных счетах;
    \item движении денежных средств.
\end{itemize}

\begin{table}[thp]
    \centering
    \caption{Сущности и их данные}
    \label{tab:data}
    \begin{tabular}{|p{7cm}|p{7cm}|}
         \hline
         \textbf{Сущность} & \textbf{Данные} \\
         \hline
         Пользователь & Имя, почта, пароль, номер телефона, паспортные данные\\
          \hline
         Товар & Название, описание, фотографии, собственник \\
          \hline
         Аукцион & Товар, организатор, дата проведения \\
         \hline
         Ставка & Сумма, дата создания \\
         \hline
         Заключенный договор & Аукцион, ставка, победитель \\
         \hline
         Номинальный счет & Банк, в котором был открыт счет; реквизиты счета \\
         \hline
         Движение денежных средств & Сумма, отправитель, получатель, статус \\
         \hline
    \end{tabular}
\end{table}

На рисунке \ref{img:er-chena} представлена диаграмма модели сущность-связь в нотации Чена. На рисунках \ref{img:state-user-form}-\ref{img:state-deal} представлены диаграммы состояний сущностей системы.

\sidewaysimgw{er-chena}{thp}{1\textwidth}{Диаграмма модели сущность-связь в нотации Чена}

\newpage

\imgs{state-user-form}{thp}{1}{Диаграмма состояний анкеты}

\imgs{state-product}{thp}{1}{Диаграмма состояний продукта}

\imgs{state-auction}{thp}{1}{Диаграмма состояний аукциона}

\newpage

\imgs{state-offer}{thp}{1}{Диаграмма состояний предложения}

\imgs{state-transaction}{thp}{1}{Диаграмма состояний транзакции}

\imgs{state-deal}{thp}{1}{Диаграмма состояний сделки}

%? рассказать про подтверждение действий по токену
%? рассказать про учет изменений информации о пользователях
%? рассказать про мультивалютность

\section{Классификация БД}
\textbf{База данных} -- совокупность данных, хранимых в соответствии со схемой данных, манипулирование которыми выполняют в соответствии с правилами средств моделирования данных. \cite{gost10032db}

\textbf{Модель данных} -- это абстрактное, самодостаточное, логическое определение объектов, операторов и прочих элементов, в совокупности составляющих абстрактную машину доступа к данным, с которой взаимодействует пользователь. Эти объекты позволяют моделировать структуру данных, а операторы — поведение данных. \cite{дейт2008введение}

% Каждая БД и СУБД строится на основе некоторой явной или неявной модели данных. Все СУБД, построенные на одной и той же модели данных, относят к одному типу. Поэтому, классификация СУБД по используемой модели данных является основной. Наиболее известными моделями являются следующие \cite{watt2014database}:

Каждая БД строится на основе некоторой явной или неявной модели данных. Все БД, построенные на одной и той же модели данных, относят к одному типу. Поэтому, классификация БД по используемой модели данных является основной. Наиболее известными моделями являются следующие \cite{watt2014database}:

\begin{itemize}
    \item \textbf{дореляционные модели данных} являются предшественниками реляционных баз данных. Наиболее известные представители:
    \begin{itemize}
        \item \textbf{сетевая модель данных} представляет данные в виде типов записей. Эта модель также представляет ограниченный тип отношения один ко многим, называемый типом набора;
        \item \textbf{иерархическая модель} представляет данные в виде иерархической древовидной структуры. Каждая ветвь иерархии представляет собой ряд связанных записей;
    \end{itemize}
    \item \textbf{реляционная модель данных} представляет данные в виде отношений или таблиц;
    \item \textbf{постреляционные модели данных} появились после реляционная модели и удовлетворяют требованиям, на которые реляционные базы данных не способны. К таким требованиям относятся: большое количество данных, огромное количество пользователей, сложные данные. Наиболее известными представителями постреляционных моделей являются:
    \begin{itemize}
        \item \textbf{документоориентированная модель данных} предназначена для хранения иерархических структур данных, которые в этой модели называются документами. Документ имеет структуру дерева, в листьях которого находятся хранимые данные. Документы можно сгруппировать в коллекции;
        \item \textbf{модель данных ключ-значение} определяет структуру хранения в виде ассоциативного массива. Это позволяет создавать системы, главной особенностью которых является быстрое извлечение данных, так как структура хранения жестко задана.
    \end{itemize}
\end{itemize}

\section{Выбор модели хранения}
Для решения поставленной задачи были выбраны следующие критерии:

\begin{itemize}
    \item \textbf{наличие ACID} необходимо для проведения торга, так как вовремя создания предложений возможно состояние гонки;
    \item \textbf{соблюдение целостности ссылочных данных} необходимо для гарантии целостности связей между сущностями системы.
\end{itemize}

В таблице \ref{tab:comparedatamodels} представлено сравнение наиболее известных моделей хранения данных.

\begin{table}[thp]
    \centering
    \caption{Сравнение моделей хранения}
    \label{tab:comparedatamodels}
    \begin{tabular}{|p{10cm}|p{1.5cm}|p{3.5cm}|}
         \hline
         \textbf{Модель хранения данных} & \textbf{ACID} & \textbf{Ссылочная \vbox{целостность}} \\
         \hline
         сетевая модель & - & - \\
         \hline
         иерархическая модель & - & + \\
         \hline
         реляционная модель & + & + \\
         \hline
         документоориентированная модель & - & - \\
         \hline
         модель ключ-значение & - & - \\
         \hline
    \end{tabular}
\end{table}

Из приведенного сравнения можно сделать вывод, что для решения поставленной задачи лучше всего подходит реляционная модель данных, к тому же формализованные данные хорошо представляются в виде отношений.

\section*{Вывод}
\addcontentsline{toc}{section}{Вывод}
В данном разделе были рассмотрены определение и  классификация аукционов, порядок проведения и правовой аспект английского аукциона. Были представлены варианты использования приложения, формализованы данные, построена диаграмма сущность-связь, рассмотрены состояния сущностей систем. Проанализированы и классифицированы модели хранения данных, выбраны и обоснованы критерии сравнения моделей, и из рассмотренных моделей была выбрана оптимальная для решения поставленной задачи.
