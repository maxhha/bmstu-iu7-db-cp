\chapter{Технологический раздел}

В данном разделе будет произведен обзор и сравнение существующих реляционных СУБД по составленным критериям и выбор СУБД, которая является оптимальной для решения поставленной задачи. Будут определены средства программной реализации и описана инициализация и система безопасности базы данных. А так же будут приведены примеры запросов к базе данных и примеры работы приложения.

\section{Обзор существующих реляционных СУБД}

\subsection{MySQL}  
% https://www.mysql.com/

MySQL\cite{mysql} -- это база данных компании Oracle, впервые выпущенная в 1995 году. Она долгое время был на вершине многих рейтинговых списков. Это связано с тем, что это одна из первых баз данных с открытым исходным кодом, наполненная полезным и надежным функционалом. Oracle также предлагает платную версию MySQL с дополнительными функциями и поддержкой. MySQL являются хорошей отправной точкой для веб-проектов, но проигрывает в случаях, когда нужны расширенные функции защиты данных и когда нужна поддержку полуструктурированными данными, такими как JSON.

\subsection{Oracle}
% https://www.oracle.com/database/

Разработанная в 1979 году, Oracle\cite{oracledb} остается популярной базой данных, особенно для СУБД корпоративного уровня. Это одна из самых зрелых и стабильных баз данных на сегодняшний день. Он используется крупнейшими компаниями из списка Fortune 500 по всему миру для своих транзакций. Она предлагает продвинутые функции как для структурированных, так и для полуструктурированных данных, поддерживает блокчейн-таблицы, облегчает мгновенные транзакции и помогает создавать как OLAP, так и OLTP в одном экземпляре базы данных. Однако у нее есть существенный недостаток по сравнению с другими базами данных: она не имеет открытого исходного кода. Стоимость запуска базы данных Oracle для предприятия с несколькими сотнями сотрудников может достигать тысяч долларов.

\subsection{PostgreSQL}
% https://www.postgresql.org

PostgreSQL\cite{postgresql} была разработана в Калифорнийском университете в Беркли и выпущена в 1989 году. Изначальное название POSTGRES является преемником базы данных INGRES. В 1996 году его название стало PostgreSQL, чтобы указать на поддержку SQL. По инновациям и возможностям PostgreSQL, несомненно, находится на первом месте. PostgreSQL обрабатывает полуструктурированные данные, такие как JSON, и имеет поддержку распределенного SQL. Последнее полезно при работе с миллионами транзакций в Интернете. PostgreSQL имеет является СУБД с открытым исходным кодом и работает во всех основных операционных системах, таких как Windows, MacOS, а также в системах семейства Unix. У нее есть большое сообщество пользователей, которые разрабатывают плагины и библиотеки. Более того имеется возможность писать скрипты на Python и запускать их в PostgreSQL.

% Microsoft SQL
% SQLite
% MariaDB

% https://www.keycdn.com/blog/popular-databases

\section{Выбор СУБД}
Для выбора СУБД были составлены следующие критерии:
\begin{itemize}
    \item \textbf{поддержка JSON} нужна для реализации сущности \textit{Токен}. Токеном подтверждаются разные действия пользователя, для которых нужны свои данные, структура которых заранее не определена; 
    \item \textbf{бесплатное использование} СУБД необходимо для минимизации затрат на разработку системы.
\end{itemize}

Сравнение наиболее известных СУБД представлено в таблице \ref{tab:comparedbms}.

\begin{table}[th]
    \centering
    \caption{Сравнение СУБД}
    \label{tab:comparedbms}
    \begin{tabular}{|c|c|c|}
         \hline
         \textbf{СУБД} & \textbf{поддержка JSON} & \textbf{Бесплатное использование} \\
         \hline
         MySQL & - & + \\
         \hline
         Oracle & + & - \\
         \hline
         PostgreSQL & + & + \\
         \hline
    \end{tabular}
\end{table}

Из приведенного сравнения можно сделать вывод, что для реализации поставленной задачи лучше всего подходит PostgreSQL, так как является СУБД с открытым исходным кодом и позволяет обрабатывает JSON.

\section{Средства реализации}

Для взаимодействия с базой данных необходимо написать приложение. Приложение было спроектировано в разделе \ref{lbl:server-arch}. Для его реализации были выбраны следующие технологии:
\begin{itemize}
    \item язык программирования Go\cite{golang}, как наиболее подходящий для написания веб-серверов;
    \item Gin Web Framework\cite{gingonic} -- HTTP фреймворк;
    \item gqlgen\cite{gqlgen} -- библиотека для написания GraphQL\cite{graphql} серверов;
    \item GORM\cite{gorm} -- ORM, написанная на Go;
    \item Docker Compose\cite{docker} позволяет развернуть и запустить отдельные части сервера. 
\end{itemize}

\section{Инициализация базы данных}

Для инициализации базы данных использовался специальный скрипты миграций. Каждый скрипт писался по одному шаблону: сначала идет проверка наличия идентификатора миграции в таблице миграций, и при его отсутствии, применяется тело миграции. В конце этого тела идентификатор миграции добавляется в таблицу миграций. Весь скрипт миграции происходит одной транзакцией и завершается оператором \texttt{COMMIT}. В начале первого скрипта миграции - инициализации базы данных - происходит создание таблицы миграций.

В приложениях \ref{apdx:init} и \ref{apdx:db-users} приведены скрипты инициализации базы данных и создания пользователей уровня базы данных соответственно.

Для развертки базы данных используется \texttt{Dockerfile}, код которого приведен на листинге \ref{lst:Dockerfile.db}. Помимо самой базы данных \texttt{Postgres} в контейнер помещается расширение \texttt{pg\_cron}\cite{pgcron}, которое позволяет периодично запускать задачи, такие как удаление гостей из таблицы пользователей.

\section{Запросы к базе данных}

На листингах \ref{lst:queries/select-available.sql} и \ref{lst:queries/select-products.sql} приведены сложные запросы к базе данных, использованные в сервере. В приложении \ref{apdx:tr-offer} представлена последовательность запросов с использованием транзакции.

\newpage

\listingfile{Dockerfile.db}{dockerfile}{Конфигурация контейнера с базой данных}{}

\newpage

\listingfile{queries/select-available.sql}{sql}{Получение свободных средств на счете}{}

\newpage

\listingfile{queries/select-products.sql}{sql}{Получение товаров, владельцем которых является определенный пользователь}{}

\section{Система безопасности сервера БД}

Безопасность сервера БД обеспечивается с помощью ролевой модели на уровне базы данных. Выделенные роли базы данных представлены в таблице~\ref{tab:db-roles}. В приложении~\ref{apdx:db-users} представлен скрипт создания ролей.

\begin{table}[!th]
    \centering
    \caption{Роли базы данных}
    \label{tab:db-roles}
    \begin{tabular}{|p{3cm}|p{12cm}|}
         \hline
         \textbf{Роль} & \textbf{Права} \\
         \hline
         \texttt{server} & чтение, добавление и обновление всех таблиц \\
         \hline
         \texttt{bankgate} & чтение, добавление и обновление транзакций \\
         \hline
         \texttt{dealer} & чтение и обновление аукционов и предложений; чтение, добавление и обновление истории сделок \\
         \hline
         \texttt{viewer} & чтение всех таблиц \\
         \hline
    \end{tabular}
\end{table}

\section{Примеры работы}

API сервера можно использовать при помощи любого инструмента, позволяющего отправлять HTTP запросы. Для приведенных примеров работы \ref{lst:usage/step-1.txt}-\ref{lst:usage/step-4.txt} была использована команда \texttt{gq} из библиотеки \texttt{graphqlurl}\cite{graphqlurl}. Она позволяет делать запросы к GraphQL серверу в одной строке и ее интерфейс аналогичен \texttt{curl}\cite{curl}.

\listingfile{usage/step-1.txt}{bash}{Регистрация гостя}{}

\newpage

\listingfile{usage/step-2.txt}{bash}{Привязка пользователя к почте}{}

\newpage

\listingfile{usage/step-3.txt}{bash}{Вход по почте и паролю}{}

\listingfile{usage/step-4.txt}{bash}{Получение черновика анкеты текущего пользователя}{}

\section*{Вывод}
\addcontentsline{toc}{section}{Вывод}

В данном разделе был произведен обзор и сравнение существующих реляционных СУБД по составленным критериям, выбрана СУБД, которая является оптимальной для решения поставленной задачи. Были определены средства программной реализации и описана инициализация и система безопасности базы данных. А так же были приведены примеры запросов к базе данных и примеры работы приложения.